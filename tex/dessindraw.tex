% arara: pdflatex
% arara: bibtex
% arara: pdflatex

\documentclass{amsart}

%%%%%%%%%%%%%%%%%%%%%%%%%%%%%%%%%%%%%%%%%%%%%%%%%%%%%
% BELYIDB
%%%%%%%%%%%%%%%%%%%%%%%%%%%%%%%%%%%%%%%%%%%%%%%%%%%%%

%\usepackage[utf8]{inputenc}
%\usepackage[OT2,T1]{fontenc}
%\usepackage[english]{babel}

%\usepackage{amssymb,amsthm,amsmath,amsxtra}
%% \usepackage[all]{xy}
%%\usepackage{fullpage}
%\usepackage{comment}
%\usepackage{graphicx}
%\usepackage{hyperref}  % incompatible with showkeys
%\hypersetup{colorlinks=true,urlcolor=blue,citecolor=blue,linkcolor=blue}
%% \usepackage[hypcap]{caption}
%% \usepackage[notcite]{showkeys}	
%% \usepackage[margin = 1.5in]{geometry}
%% \usepackage[linesnumbered]{algorithm2e}
%\usepackage{color}
%\usepackage{array}
%\usepackage{diagbox}
%\usepackage{colonequals}

%\usepackage{placeins} % for \FloatBarrier command

%\usepackage[dvipsnames]{xcolor}

%\usepackage{chngcntr}
%\usepackage{booktabs}
%\usepackage{mathrsfs}
%%\usepackage{pstricks}
%%\usepackage{pst-node,pst-tree}
%%\usepackage{pstricks-add}
%%\usepackage{auto-pst-pdf}

%\numberwithin{equation}{subsection}
%\renewcommand{\thefigure}{\arabic{figure}}

%\theoremstyle{plain}
%\newtheorem{thm}[equation]{Theorem}
%\newtheorem{prop}[equation]{Proposition}
%\newtheorem{lem}[equation]{Lemma}
%\newtheorem{cor}[equation]{Corollary}
%\newtheorem{prob}[equation]{Problem}
%\newtheorem{ques}[equation]{Question}
%\newtheorem{conj}[equation]{Conjecture}
%\newtheorem*{cor*}{Corollary}
%\newtheorem*{prob*}{Problem}
%\newtheorem*{thm*}{Theorem}
%\newtheorem*{thma*}{Theorem A}
%\newtheorem*{thmb*}{Theorem B}
%\newtheorem*{thmc*}{Theorem C}

%\theoremstyle{definition}
%\newtheorem{defn}[equation]{Definition}
%\newtheorem{alg}[equation]{Algorithm}
%\newtheorem{exm}[equation]{Example}

%\theoremstyle{remark}
%\newtheorem{rmk}[equation]{Remark}

%\newenvironment{enumalph}
%{\begin{enumerate}\renewcommand{\labelenumi}{\textnormal{(\alph{enumi})}}}
%{\end{enumerate}}

%\newenvironment{enumalgalph}
%{\begin{enumerate}\renewcommand{\labelenumii}{\alph{enumii}.}}
%{\end{enumerate}}

%\newenvironment{enumroman}
%{\begin{enumerate}\renewcommand{\labelenumi}{\textnormal{(\roman{enumi})}}}
%{\end{enumerate}}

%\newenvironment{enumalg}
%{\begin{enumerate}\renewcommand{\labelenumi}{\arabic{enumi}.}}
%{\end{enumerate}}
%\usepackage[inline, shortlabels]{enumitem}
%%\usepackage{enumerate}
%\newcommand{\defi}[1]{\textit{\textsf{#1}}} % for defined terms


%\setlength{\hfuzz}{4pt}

%\DeclareMathOperator{\arccosh}{arccosh}
%\DeclareMathOperator{\Aut}{Aut}
%\DeclareMathOperator{\Cl}{Cl}
%\DeclareMathOperator{\cyc}{cyc}
%\DeclareMathOperator{\ddiv}{div}
%\DeclareMathOperator{\disc}{disc}
%\DeclareMathOperator{\End}{End}
%\DeclareMathOperator{\Frob}{Frob}
%\DeclareMathOperator{\Gal}{Gal}
%\DeclareMathOperator{\id}{id}
%\DeclareMathOperator{\impart}{Im}
%\DeclareMathOperator{\Inn}{Inn}
%\DeclareMathOperator{\Isom}{Isom}
%\DeclareMathOperator{\M}{M}
%\DeclareMathOperator{\Mon}{Mon}
%\DeclareMathOperator{\N}{N}
%\DeclareMathOperator{\opdiv}{div}
%\DeclareMathOperator{\nrd}{nrd}
%\DeclareMathOperator{\ord}{ord}
%\DeclareMathOperator{\Out}{Out}
%\DeclareMathOperator{\PGL}{PGL}
%\DeclareMathOperator{\PSL}{PSL}
%\DeclareMathOperator{\Real}{Re}
%\DeclareMathOperator{\repart}{Re}
%\DeclareMathOperator{\SO}{SO}
%\DeclareMathOperator{\Spec}{Spec}
%\DeclareMathOperator{\Stab}{Stab}
%\DeclareMathOperator{\SL}{SL}
%\DeclareMathOperator{\GL}{GL}
%\DeclareMathOperator{\tr}{tr}
%\DeclareMathOperator{\(P)SL}{(P)SL}
%\DeclareMathOperator{\PSU}{PSU}

%\let\div\relax
%\DeclareMathOperator{\div}{div}

%\newcommand{\C}{\mathbb C}
%\newcommand{\F}{\mathbb F}
%\newcommand{\HH}{\mathbb H}
%\newcommand{\PP}{\mathbb P}
%\newcommand{\Q}{\mathbb Q}
%\newcommand{\R}{\mathbb R}
%\newcommand{\Z}{\mathbb Z}
%\newcommand{\Qbar}{{\mathbb Q}^{\textup{al}}}
%\newcommand{\Qal}{{\mathbb Q}^{\textup{al}}}
%\newcommand{\kbar}{\overline{k}}
%\newcommand{\Qab}{{\mathbb Q}_{\textup{ab}}}

%\newcommand{\T}{\mathcal T}
%\renewcommand{\L}{\mathscr L}
%\DeclareMathOperator{\Princ}{Princ}

%\newcommand{\newt}{\mu}

%% \newcommand{\Belyi}{Bely\u{\i}}
%\newcommand{\Belyi}{Belyi}

%\newcommand{\fraka}{\mathfrak{a}}
%\newcommand{\frakd}{\mathfrak{d}}
%\newcommand{\frakn}{\mathfrak{n}}
%\newcommand{\frakM}{\mathfrak{M}}
%\newcommand{\frakN}{\mathfrak{N}}
%\newcommand{\frakp}{\mathfrak{p}}
%\newcommand{\frakP}{\mathfrak{P}}
%\newcommand{\frakq}{\mathfrak{q}}

%\newcommand{\calD}{\mathcal{D}}
%\newcommand{\calG}{\mathcal{G}}
%\newcommand{\calH}{\mathcal{H}}
%\newcommand{\calM}{\mathcal{M}}
%\newcommand{\calO}{\mathcal{O}}
%\newcommand{\calP}{\mathcal{P}}

%\newcommand{\psmod}[1]{~(\textup{\text{mod}}~{#1})}
%\newcommand{\legen}[2]{\left(\frac{#1}{#2}\right)}

%\newcommand{\quat}[2]{\displaystyle{\biggl(\frac{#1}{#2}\biggr)}}

%\newcommand{\la}{\langle}
%\newcommand{\ra}{\rangle}

%\newcommand{\gunder}{\underline{g}}
%\newcommand{\tunder}{\underline{t}}
%\newcommand{\Cunder}{\underline{C}}

%\DeclareMathOperator{\opchar}{char}

%\newcommand{\Deltabar}{\overline{\Delta}}
%\newcommand{\gammabar}{\overline{\gamma}}
%\newcommand{\Gammabar}{\overline{\Gamma}}
%\newcommand{\Fbar}{\overline{F}}
%\newcommand{\Bbar}{\overline{B}}
%\newcommand{\calObar}{\overline{\mathcal{O}}}
%\newcommand{\frakpbar}{\overline{\mathfrak{p}}}
%\newcommand{\frakdbar}{\overline{\mathfrak{d}}}

%\newcommand{\zetabar}{\overline{\zeta}}

%\newcommand{\sbwk}{{}_{\textup{wkrat}}}
%\newcommand{\sbnum}{{}_{\textup{num}}}
%\newcommand{\sbrat}{{}_{\textup{rat}}}
%\newcommand{\sbtr}{{}_{\textup{tr}}}
%\newcommand{\sbtors}{{}_{\textup{tors}}}
%\newcommand{\lcm}{\operatorname{lcm}}
%\newcommand{\ns}{\operatorname{ns}}

%\newcommand{\prodprime}{\sideset{}{^{'}}{\prod}}

%\newcommand{\Bhat}{\widehat{B}}
%\newcommand{\Fhat}{\widehat{F}}
%\newcommand{\PGamma}{\mathrm{P}\Gamma}
%\newcommand{\calOhat}{\widehat{\mathcal O}}
%\newcommand{\gammahat}{\widehat{\gamma}}
%\newcommand{\alphahat}{\widehat{\alpha}}
%\newcommand{\ahat}{\widehat{a}}
%\newcommand{\Zhat}{\widehat{\mathbb Z}}
%\newcommand{\ZFhat}{\widehat{\mathbb Z}_F}

%\newcommand{\Part}{\mathrm{Part}}

%\newcommand{\jv}[1]{{\color{red} \sf JV: [#1]}}

%\newcommand{\sss}[1]{{\color{green} \sf S$^3$: [#1]}}
%\newcommand{\mm}[1]{{\color{blue} \sf MM: [#1]}}
%\newcommand{\mk}[1]{{\color{orange} \sf MK: [#1]}}

%\newcommand{\jrs}[1]{{\color{cyan} \sf JRS: [#1]}}

%\setcounter{tocdepth}{1}

%\everymath{\displaystyle}

%%%%%%%%%%%%%%%%%%%%%%%%%%%%%%%%%%%%%%%%%%%%%%%%%%
% Additional Packages (add as desired)
%%%%%%%%%%%%%%%%%%%%%%%%%%%%%%%%%%%%%%%%%%%%%%%%%%
\usepackage{amsmath,amssymb,amsthm,amsxtra}
\usepackage{algorithm}
% \usepackage{showkeys}
\usepackage{chngcntr}
\usepackage{mathrsfs} 
\usepackage{lipsum}
\usepackage{xcolor}
\usepackage{hyperref}
\hypersetup{colorlinks=true,urlcolor=blue,citecolor=blue,linkcolor=blue}
\usepackage{tikz}
\usepackage{colonequals}
% \usepackage{caption}
\usepackage{enumerate}
\usepackage{booktabs}
\usepackage{longtable}
\usetikzlibrary{arrows,calc,automata,shadows,backgrounds,positioning,intersections,fadings,decorations.pathreplacing,shapes,matrix}
\usepackage{tikz-cd}
\tikzset{commutative diagrams/.cd, arrow style = tikz, diagrams = {>=latex}}
\tikzset{>=latex}
\usepackage{filecontents}
\usepackage{pgfplots, pgfplotstable}
\usepgfplotslibrary{statistics}

%%%%%% begin TCOLORBOX
\usepackage{tcolorbox, listings}
% \lstdefinestyle{mystyle}{
%   basicstyle=\ttfamily
% %  basicstyle=\ttfamily,
% %  numbers=left,
% %  numberstyle=\tiny,
% %  numbersep=5pt
% }
\tcbuselibrary{listings, skins, breakable}
\newtcblisting{magma}{%
  title={\textsf{Magma}},
  %every listing line={\textcolor{blue}{\texttt{> }}},
  arc=0mm,
  top=0mm,
  bottom=0mm,
  left=3mm,
  right=0mm,
  width=\textwidth,
  boxrule=0.5pt,
  colback=gray!20,
  %spartan,
  listing only,
  % listing options={style=mystyle},
  breakable
}
\newtcblisting{code}{%
  title={\textsf{Shell}},
  %every listing line={\textcolor{blue}{\texttt{> }}},
  arc=0mm,
  top=0mm,
  bottom=0mm,
  left=3mm,
  right=0mm,
  width=\textwidth,
  boxrule=0.5pt,
  colback=gray!20,
  %spartan,
  listing only,
  % listing options={style=mystyle},
  breakable
}
%%%%%% end TCOLORBOX

%\usepackage{index} %Uncomment if you would like to have an index. Compiling with an index takes more work than compiling without one. You will have to look up how to use the index package.

%%%%%%%%%%%%%%%%%%%%%%%%%%%%%%%%%%%%%%%%%%%%%%%%%%
%Formatting
%%%%%%%%%%%%%%%%%%%%%%%%%%%%%%%%%%%%%%%%%%%%%%%%%%
%  Change or add to as desired. 
%  These two commands make the first enumerations look like (a)
%  And the second level like (i).  
% \renewcommand{\labelenumi}{(\alph{enumi})}
% \renewcommand{\labelenumii}{(\roman{enumii})}
%  These commands make the section headings Boldface and not all
%  caps. It also removes the chapter numbers  
% \renewcommand{\chaptermark}[1]{\markboth{{\sc #1}}{{\sc #1}}}
\renewcommand{\sectionmark}[1]{\markright{{\sc \thesection\ #1}}}
%  These commands have lowercase headings with chapter numbers. 
%\renewcommand{\chaptermark}[1]{markboth{#1}{}}
%\renewcommand{\sectionmark}[1]{\markright{\thesection\ #1}} 
\newcommand{\OO}{\mathcal O}
\newcommand{\PP}{\mathbb P}
\newcommand{\CC}{\mathbb C}
\newcommand{\RR}{\mathbb R}
\newcommand{\QQ}{\mathbb Q}
\newcommand{\ZZ}{\mathbb Z}
\renewcommand{\AA}{\mathbb A}
\newcommand{\defi}[1]{\textsf{#1}}
\newcommand{\jv}[1]{{\color{red} \sf JV: [#1]}}
\newcommand{\mm}[1]{{\color{blue} \sf MM: [#1]}}
\newcommand{\wt}[1]{\widetilde{#1}}
\newcommand{\QQal}{{\mathbb Q}^{\textup{al}}}
\newcommand{\FFqal}{{\mathbb F}_q^{\textup{al}}}
\newcommand{\QQab}{{\mathbb Q}^{\textup{ab}}}
\newcommand{\QQbar}{{\mathbb Q}^{\textup{al}}}
\newcommand{\Kal}{{{K}^{\textup{al}}}}
\newcommand{\Kab}{{K}^{\textup{ab}}}
\newcommand{\kbar}{\overline{k}}
\newcommand{\Kbar}{\overline{K}}
\newcommand{\LL}{\mathscr L}
\newcommand{\sm}{\setminus}
\newcommand{\FF}{\mathbb{F}}
\renewcommand{\ker}{\operatorname{ker}}

\DeclareMathOperator{\con}{con}
\DeclareMathOperator{\Div}{Div}
\DeclareMathOperator{\Princ}{Princ}
\DeclareMathOperator{\Pic}{Pic}
\DeclareMathOperator{\ddiv}{div}
\DeclareMathOperator{\sat}{sat}
\DeclareMathOperator{\ddeg}{deg}
\DeclareMathOperator{\ddim}{dim}
\DeclareMathOperator{\rred}{red}
% \renewcommand{\rred}{\operatorname{red}}
\DeclareMathOperator{\Lifts}{Lifts}
\DeclareMathOperator{\order}{order}
\DeclareMathOperator{\Aut}{Aut}
\DeclareMathOperator{\PGL}{PGL}
\DeclareMathOperator{\Mon}{Mon}
\DeclareMathOperator{\Gal}{Gal}
\DeclareMathOperator{\Inn}{Inn}
\DeclareMathOperator{\Out}{Out}
\DeclareMathOperator{\supp}{supp}
\DeclareMathOperator{\ord}{ord}
\DeclareMathOperator{\mult}{mult}
\DeclareMathOperator{\stab}{stab}
\DeclareMathOperator{\orb}{orb}
\DeclareMathOperator{\id}{id}
% \DeclareMathOperator{\ker}{ker}
\DeclareMathOperator{\GL}{GL}
\DeclareMathOperator{\charpoly}{charpoly}
% \DeclareMathOperator{\det}{det}
\DeclareMathOperator{\Tr}{tr}
\DeclareMathOperator{\Pl}{Pl}
\DeclareMathOperator{\Cl}{Cl}
\DeclareMathOperator{\Jac}{Jac}

%%%%%%%%%%%%%%%%%%%%%%%%%%%%%%%%%%%%%%%%%%%
%numberwithin section for
%tables equations and figures
%%%%%%%%%%%%%%%%%%%%%%%%%%%%%%%%%%%%%%%%%%%
\numberwithin{equation}{section}
% \renewcommand{\thefigure}{\arabic{chapter}.\arabic{section}.\arabic{equation}}
\renewcommand{\thefigure}{\arabic{section}.\arabic{equation}}
% \numberwithin{equation}{section}
% \numberwithin{figure}{section}
% \numberwithin{table}{section}
% \counterwithin{equation}{section}
% \counterwithin{figure}{section}
% \counterwithin{table}{section}
% \renewcommand{\thefigure}{\arabic{figure}}
%%%%%%%%%
\makeatletter
\let\c@equation\c@figure
\makeatother
% A basic set of theorem declarations.  Add or remove as desired. 
% \newtheorem{prop}{Proposition}[chapter]
\newtheorem{theorem}[equation]{Theorem}
\newtheorem{prop}[equation]{Proposition}
\newtheorem{conj}[equation]{Conjecture}
\newtheorem{lemma}[equation]{Lemma}
\newtheorem{corr}[equation]{Corollary}
\theoremstyle{definition}
\newtheorem{definition}[equation]{Definition}
\newtheorem{alg}[equation]{Algorithm}
\newtheorem{notation}[equation]{Notation}
\theoremstyle{remark}
\newtheorem{remark}[equation]{Remark}
\newtheorem{example}[equation]{Example}
\newtheorem*{claim}{Claim}
\newtheorem{question}[equation]{Question}

\begin{document}

\title{How to draw a dessin?}

\author{Michael Musty}
\address{ICERM Fall 2019}
\email{michaelmusty@gmail.com}

\date{\today}

\begin{abstract}
  In Seciton \ref{sec:background}
  we review some background about dessins d'enfants,
  Belyi maps, and some of the reasons the reader
  might be interested in these objects.
  In Section \ref{sec:catalogs}
  we review the existing methods for drawing
  pictures of dessins d'enfants and the available
  catalogs in the literature.
\end{abstract}

\maketitle
\tableofcontents

\section{Background}{\label{sec:background}
  \subsection{Equivalent categories}{
    \subsection{Dessins}{
    }
    \subsection{Clean dessins}{
    }
    \subsection{Ribbon graphs}{
    }
    \subsection{A zoo of bijections}{
    }
  }
}
\section{Existing catalogs}{\label{sec:catalogs}
}
\section{Drawing techniques}{
}
\section{Degree $\leq 4$}{
}
\section{Degree $5$}{
}
\section{Degree $6$}{
}
\section{Degree $7$}{
}
\section{Degree $8$}{
}

\nocite{*}
\bibliographystyle{amsplain}
\bibliography{dessindraw}

%\begin{thebibliography}{999}

%\bibitem{Adrianov}
%N.~M.~Adrianov, N.~Ya.~Amburg, V.~A.~Dremov, Yu.~A.~Levitskaya, E.~M.~Kreines, Yu.~Yu.~Kochetkov, V.~F.~Nasretdinova, G.~B.~Shabat, \emph{Catalog of dessins d'enfants with $\leq 4$ edges}, \verb|arXiv:0710.2658v1|, 14 October 2007.

%% \bibitem{AdrianovZvonkin}
%% Nikolai Adrianov and Alexander Zvonkin, \emph{Weighted trees with primitive edge rotation groups}, 2015, preprint.

%\bibitem{Belyi}
%G.V.~\Belyi, \emph{Galois extensions of a maximal cyclotomic field}, Math.\ USSR-Izv.\ \textbf{14} (1980), no.~2, 247--256.

%%\bibitem{Belyi2}
%% G.V.~\Belyi, \emph{A new proof of the three-point theorem}, translation in Sb.\ Math.\ \textbf{193} (2002), no.\ 3--4, 329--332.

%\bibitem{Betrema}
%Jean B\'etr\'ema, Danielle P\'er\'e, Alexander Zvonkin, \textit{Plane trees and their Shabat polynomials}, Catalog (5th ed.) Publication du LaBRI No. 92-75 (1992).

%\bibitem{Birch}
%Bryan Birch, \emph{Noncongruence subgroups, covers and drawings}, The Grothendieck theory of dessins d'enfants (ed.\ Leila Schneps), London Math.\ Soc.\ Lecture Note Ser., vol.\ 200, Cambridge Univ.\ Press, Cambridge, 1994, 25--46.

%\bibitem{bose}
%S. ~Bose, J. ~Gundry, Y. ~He, \emph{Gauge theories and dessins d'enfants: beyond the torus}, Journal of High Energy Physics, (2015), no.~1, 135.

%\bibitem{Magma}
%W.~Bosma, J.~Cannon, and C.~Playoust, \emph{The Magma algebra system.\ I.\ The user language}, J.\ Symbolic Comput.\ \textbf{24} (3--4), 1997, 235--265.

%\bibitem{CouveignesGranboulan}
%Jean-Marc Couveignes and Louis Granboulan, \emph{Dessins from a geometric point of view}, in \emph{The Grothendieck theory of dessins d'enfants}, London Math.\ Soc.\ Lecture Note Ser., vol.\ 200, Cambridge University Press, 1994, 79--113.

%% \bibitem{DebesEmsalem}
%% P.~D\`ebes and M.~Emsalem, \emph{On fields of moduli of curves}, J.\ Algebra \textbf{211} (1999), no.~1, 42--56.

%\bibitem{Deligne}
%P.~Deligne, \emph{Le groupe fondamental de la droite projective moins trois points}, Galois groups over ${\bf Q}$ (Berkeley, CA, 1987), Math. Sci. Res. Inst. Publ. 16, Springer, 1989, 79--297.

%\bibitem{Elkin}
%A.~Elkin, \emph{Belyi-Maps}, \url{https://github.com/arsenelkin/Belyi-Maps}

%% \bibitem{Grothendieck}
%% Alexandre Grothendieck, \emph{Sketch of a programme (translation into English)}, Geometric Galois Actions. 1. Around Grothendieck's Esquisse d'un Programme, eds.\ Leila Schneps and Pierre Lochak, London Math.\ Soc.\ Lect.\ Note Series, vol.\ 242, Cambridge University Press, Cambridge, 1997, 243--283.

%\bibitem{JV}
%Ariyan Javanpeykar and John Voight, \emph{The Belyi degree of a curve is computable}, preprint.

%\bibitem{KMSV}
%M. Klug, M. Musty, S. Schiavone, and J. Voight, \emph{Numerical calculation of three-point branched covers of the projective line}, LMS J.~Comput.~Math.\ \textbf{17} (2014), no.~1, 379--430.

%% \bibitem{LLL}
%% Arjen K.~Lenstra, Hendrik W.~Lenstra, Jr. and L\'azl\'o Lov\' asz, \emph{Factoring polynomials with rational coefficients}, Math. Ann.~\textbf{261} (1982), 513--534.

%\bibitem{Malle} Gunter Malle, \emph{Fields of definition of some three point
%  ramified field extensions}, in \emph{The Grothendieck theory of dessins d'enfants}, London Math.\ Soc.\ Lecture Note Ser., vol.\
%  200, Cambridge University Press, 1994, 147--168.
  
%\bibitem{Monien}
%Hartmut Monien, \emph{The sporadic group $J2$, Hauptmodul and Belyi map}, \texttt{arXiv:1703.05200}.

%\bibitem{Co3}
%Hartmut Monien, \emph{The sporadic group $\mathrm{Co}_3$, Hauptmodul and Belyi map}, \texttt{arXiv:1802.06923}.

%%\bibitem{Shimura1}
%%Goro Shimura, \emph{On some arithmetic properties of modular forms of one and several variables}, Ann.\ of Math.\ \textbf{102} (1975), 491--515.

%%\bibitem{Shimura2}
%% Goro Shimura, \emph{On the derivatives of theta functions and modular forms}, Duke Math.\ J.\ \textbf{44} (1977), 365--387.

%\bibitem{SijslingVoightBirch}
%Jeroen Sijsling and John Voight, \emph{On explicit descent of marked curves and maps}, Res.\ Number Theory 2 (2016), Art.~27, 35 pp.

%\bibitem{SijslingVoight}
%Jeroen Sijsling and John Voight, \emph{On computing \Belyi\ maps}, Publ.~Math.~Besan\c{c}on: Alg\`ebre Th\'eorie Nr.~2014/1, Presses Univ.\ Franche-Comt\'e, Besan\c{c}on, 73--131.

%% \bibitem{vHV2}
%% Mark van Hoeij and Raimundas Vidunas, \emph{Belyi functions for hyperbolic hypergeometric-to-Heun transformations}, J.\ Algebra, vol.~441, 2015, 609--659.

%\bibitem{Weil}
%Andr\'e Weil, \emph{The field of definition of a variety}, Amer.\ J.\ Math.\ \textbf{78} (1956), 509--524.

%% \bibitem{Zograf}
%% Peter Zograf, \emph{Enumeration of Grothendieck's dessins and KP hierarchy}, International Mathematics Research Notices
%%Oxford University Press \textbf{24} (2015), 13533--13544.

%\end{thebibliography}

\end{document}
