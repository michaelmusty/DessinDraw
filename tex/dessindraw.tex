% arara: pdflatex
% arara: bibtex
% arara: pdflatex

\documentclass{amsart}

\usepackage[utf8]{inputenc}
\usepackage[OT2,T1]{fontenc}
\usepackage[english]{babel}

\usepackage{amssymb,amsthm,amsmath,amsxtra}
% \usepackage[all]{xy}
%\usepackage{fullpage}
\usepackage{comment}
\usepackage{graphicx}
\usepackage{hyperref}  % incompatible with showkeys
% \usepackage[hypcap]{caption}
% \usepackage[notcite]{showkeys}	
% \usepackage[margin = 1.5in]{geometry}
% \usepackage[linesnumbered]{algorithm2e}
\usepackage{color}
\usepackage{array}
\usepackage{diagbox}
\usepackage{colonequals}

\usepackage{placeins} % for \FloatBarrier command

\usepackage[dvipsnames]{xcolor}

\usepackage{chngcntr}
\usepackage{booktabs}
\usepackage{mathrsfs}
%\usepackage{pstricks}
%\usepackage{pst-node,pst-tree}
%\usepackage{pstricks-add}
%\usepackage{auto-pst-pdf}

\numberwithin{equation}{subsection}
\renewcommand{\thefigure}{\arabic{figure}}

\theoremstyle{plain}
\newtheorem{thm}[equation]{Theorem}
\newtheorem{prop}[equation]{Proposition}
\newtheorem{lem}[equation]{Lemma}
\newtheorem{cor}[equation]{Corollary}
\newtheorem{prob}[equation]{Problem}
\newtheorem{ques}[equation]{Question}
\newtheorem{conj}[equation]{Conjecture}
\newtheorem*{cor*}{Corollary}
\newtheorem*{prob*}{Problem}
\newtheorem*{thm*}{Theorem}
\newtheorem*{thma*}{Theorem A}
\newtheorem*{thmb*}{Theorem B}
\newtheorem*{thmc*}{Theorem C}

\theoremstyle{definition}
\newtheorem{defn}[equation]{Definition}
\newtheorem{alg}[equation]{Algorithm}
\newtheorem{exm}[equation]{Example}

\theoremstyle{remark}
\newtheorem{rmk}[equation]{Remark}

\newenvironment{enumalph}
{\begin{enumerate}\renewcommand{\labelenumi}{\textnormal{(\alph{enumi})}}}
{\end{enumerate}}

\newenvironment{enumalgalph}
{\begin{enumerate}\renewcommand{\labelenumii}{\alph{enumii}.}}
{\end{enumerate}}

\newenvironment{enumroman}
{\begin{enumerate}\renewcommand{\labelenumi}{\textnormal{(\roman{enumi})}}}
{\end{enumerate}}

\newenvironment{enumalg}
{\begin{enumerate}\renewcommand{\labelenumi}{\arabic{enumi}.}}
{\end{enumerate}}
\usepackage[inline, shortlabels]{enumitem}
%\usepackage{enumerate}
\newcommand{\defi}[1]{\textit{\textsf{#1}}} % for defined terms


\setlength{\hfuzz}{4pt}

\DeclareMathOperator{\arccosh}{arccosh}
\DeclareMathOperator{\Aut}{Aut}
\DeclareMathOperator{\Cl}{Cl}
\DeclareMathOperator{\cyc}{cyc}
\DeclareMathOperator{\ddiv}{div}
\DeclareMathOperator{\disc}{disc}
\DeclareMathOperator{\End}{End}
\DeclareMathOperator{\Frob}{Frob}
\DeclareMathOperator{\Gal}{Gal}
\DeclareMathOperator{\id}{id}
\DeclareMathOperator{\impart}{Im}
\DeclareMathOperator{\Inn}{Inn}
\DeclareMathOperator{\Isom}{Isom}
\DeclareMathOperator{\M}{M}
\DeclareMathOperator{\Mon}{Mon}
\DeclareMathOperator{\N}{N}
\DeclareMathOperator{\opdiv}{div}
\DeclareMathOperator{\nrd}{nrd}
\DeclareMathOperator{\ord}{ord}
\DeclareMathOperator{\Out}{Out}
\DeclareMathOperator{\PGL}{PGL}
\DeclareMathOperator{\PSL}{PSL}
\DeclareMathOperator{\Real}{Re}
\DeclareMathOperator{\repart}{Re}
\DeclareMathOperator{\SO}{SO}
\DeclareMathOperator{\Spec}{Spec}
\DeclareMathOperator{\Stab}{Stab}
\DeclareMathOperator{\SL}{SL}
\DeclareMathOperator{\GL}{GL}
\DeclareMathOperator{\tr}{tr}
\DeclareMathOperator{\(P)SL}{(P)SL}
\DeclareMathOperator{\PSU}{PSU}

\let\div\relax
\DeclareMathOperator{\div}{div}

\newcommand{\C}{\mathbb C}
\newcommand{\F}{\mathbb F}
\newcommand{\HH}{\mathbb H}
\newcommand{\PP}{\mathbb P}
\newcommand{\Q}{\mathbb Q}
\newcommand{\R}{\mathbb R}
\newcommand{\Z}{\mathbb Z}
\newcommand{\Qbar}{{\mathbb Q}^{\textup{al}}}
\newcommand{\Qal}{{\mathbb Q}^{\textup{al}}}
\newcommand{\kbar}{\overline{k}}
\newcommand{\Qab}{{\mathbb Q}_{\textup{ab}}}

\newcommand{\T}{\mathcal T}
\renewcommand{\L}{\mathscr L}
\DeclareMathOperator{\Princ}{Princ}

\newcommand{\newt}{\mu}

% \newcommand{\Belyi}{Bely\u{\i}}
\newcommand{\Belyi}{Belyi}

\newcommand{\fraka}{\mathfrak{a}}
\newcommand{\frakd}{\mathfrak{d}}
\newcommand{\frakn}{\mathfrak{n}}
\newcommand{\frakM}{\mathfrak{M}}
\newcommand{\frakN}{\mathfrak{N}}
\newcommand{\frakp}{\mathfrak{p}}
\newcommand{\frakP}{\mathfrak{P}}
\newcommand{\frakq}{\mathfrak{q}}

\newcommand{\calD}{\mathcal{D}}
\newcommand{\calG}{\mathcal{G}}
\newcommand{\calH}{\mathcal{H}}
\newcommand{\calM}{\mathcal{M}}
\newcommand{\calO}{\mathcal{O}}
\newcommand{\calP}{\mathcal{P}}

\newcommand{\psmod}[1]{~(\textup{\text{mod}}~{#1})}
\newcommand{\legen}[2]{\left(\frac{#1}{#2}\right)}

\newcommand{\quat}[2]{\displaystyle{\biggl(\frac{#1}{#2}\biggr)}}

\newcommand{\la}{\langle}
\newcommand{\ra}{\rangle}

\newcommand{\gunder}{\underline{g}}
\newcommand{\tunder}{\underline{t}}
\newcommand{\Cunder}{\underline{C}}

\DeclareMathOperator{\opchar}{char}

\newcommand{\Deltabar}{\overline{\Delta}}
\newcommand{\gammabar}{\overline{\gamma}}
\newcommand{\Gammabar}{\overline{\Gamma}}
\newcommand{\Fbar}{\overline{F}}
\newcommand{\Bbar}{\overline{B}}
\newcommand{\calObar}{\overline{\mathcal{O}}}
\newcommand{\frakpbar}{\overline{\mathfrak{p}}}
\newcommand{\frakdbar}{\overline{\mathfrak{d}}}

\newcommand{\zetabar}{\overline{\zeta}}

\newcommand{\sbwk}{{}_{\textup{wkrat}}}
\newcommand{\sbnum}{{}_{\textup{num}}}
\newcommand{\sbrat}{{}_{\textup{rat}}}
\newcommand{\sbtr}{{}_{\textup{tr}}}
\newcommand{\sbtors}{{}_{\textup{tors}}}
\newcommand{\lcm}{\operatorname{lcm}}
\newcommand{\ns}{\operatorname{ns}}

\newcommand{\prodprime}{\sideset{}{^{'}}{\prod}}

\newcommand{\Bhat}{\widehat{B}}
\newcommand{\Fhat}{\widehat{F}}
\newcommand{\PGamma}{\mathrm{P}\Gamma}
\newcommand{\calOhat}{\widehat{\mathcal O}}
\newcommand{\gammahat}{\widehat{\gamma}}
\newcommand{\alphahat}{\widehat{\alpha}}
\newcommand{\ahat}{\widehat{a}}
\newcommand{\Zhat}{\widehat{\mathbb Z}}
\newcommand{\ZFhat}{\widehat{\mathbb Z}_F}

\newcommand{\Part}{\mathrm{Part}}

\newcommand{\jv}[1]{{\color{red} \sf JV: [#1]}}

\newcommand{\sss}[1]{{\color{green} \sf S$^3$: [#1]}}
\newcommand{\mm}[1]{{\color{blue} \sf MM: [#1]}}
\newcommand{\mk}[1]{{\color{orange} \sf MK: [#1]}}

\newcommand{\jrs}[1]{{\color{cyan} \sf JRS: [#1]}}

\setcounter{tocdepth}{1}

\everymath{\displaystyle}

\begin{document}

\title{How to draw a dessin?}

\author{Michael Musty}
\address{Department of Mathematics, Dartmouth College, 6188 Kemeny Hall, Hanover, NH 03755, USA}
\email{michaelmusty@gmail.com}

\date{\today}

\begin{abstract}
\end{abstract}

\maketitle
\tableofcontents

\section{Introduction}

\nocite{*}
\bibliographystyle{amsplain}
\addcontentsline{toc}{chapter}{References}
\bibliography{dessindraw}

%\begin{thebibliography}{999}

%\bibitem{Adrianov}
%N.~M.~Adrianov, N.~Ya.~Amburg, V.~A.~Dremov, Yu.~A.~Levitskaya, E.~M.~Kreines, Yu.~Yu.~Kochetkov, V.~F.~Nasretdinova, G.~B.~Shabat, \emph{Catalog of dessins d'enfants with $\leq 4$ edges}, \verb|arXiv:0710.2658v1|, 14 October 2007.

%% \bibitem{AdrianovZvonkin}
%% Nikolai Adrianov and Alexander Zvonkin, \emph{Weighted trees with primitive edge rotation groups}, 2015, preprint.

%\bibitem{Belyi}
%G.V.~\Belyi, \emph{Galois extensions of a maximal cyclotomic field}, Math.\ USSR-Izv.\ \textbf{14} (1980), no.~2, 247--256.

%%\bibitem{Belyi2}
%% G.V.~\Belyi, \emph{A new proof of the three-point theorem}, translation in Sb.\ Math.\ \textbf{193} (2002), no.\ 3--4, 329--332.

%\bibitem{Betrema}
%Jean B\'etr\'ema, Danielle P\'er\'e, Alexander Zvonkin, \textit{Plane trees and their Shabat polynomials}, Catalog (5th ed.) Publication du LaBRI No. 92-75 (1992).

%\bibitem{Birch}
%Bryan Birch, \emph{Noncongruence subgroups, covers and drawings}, The Grothendieck theory of dessins d'enfants (ed.\ Leila Schneps), London Math.\ Soc.\ Lecture Note Ser., vol.\ 200, Cambridge Univ.\ Press, Cambridge, 1994, 25--46.

%\bibitem{bose}
%S. ~Bose, J. ~Gundry, Y. ~He, \emph{Gauge theories and dessins d'enfants: beyond the torus}, Journal of High Energy Physics, (2015), no.~1, 135.

%\bibitem{Magma}
%W.~Bosma, J.~Cannon, and C.~Playoust, \emph{The Magma algebra system.\ I.\ The user language}, J.\ Symbolic Comput.\ \textbf{24} (3--4), 1997, 235--265.

%\bibitem{CouveignesGranboulan}
%Jean-Marc Couveignes and Louis Granboulan, \emph{Dessins from a geometric point of view}, in \emph{The Grothendieck theory of dessins d'enfants}, London Math.\ Soc.\ Lecture Note Ser., vol.\ 200, Cambridge University Press, 1994, 79--113.

%% \bibitem{DebesEmsalem}
%% P.~D\`ebes and M.~Emsalem, \emph{On fields of moduli of curves}, J.\ Algebra \textbf{211} (1999), no.~1, 42--56.

%\bibitem{Deligne}
%P.~Deligne, \emph{Le groupe fondamental de la droite projective moins trois points}, Galois groups over ${\bf Q}$ (Berkeley, CA, 1987), Math. Sci. Res. Inst. Publ. 16, Springer, 1989, 79--297.

%\bibitem{Elkin}
%A.~Elkin, \emph{Belyi-Maps}, \url{https://github.com/arsenelkin/Belyi-Maps}

%% \bibitem{Grothendieck}
%% Alexandre Grothendieck, \emph{Sketch of a programme (translation into English)}, Geometric Galois Actions. 1. Around Grothendieck's Esquisse d'un Programme, eds.\ Leila Schneps and Pierre Lochak, London Math.\ Soc.\ Lect.\ Note Series, vol.\ 242, Cambridge University Press, Cambridge, 1997, 243--283.

%\bibitem{JV}
%Ariyan Javanpeykar and John Voight, \emph{The Belyi degree of a curve is computable}, preprint.

%\bibitem{KMSV}
%M. Klug, M. Musty, S. Schiavone, and J. Voight, \emph{Numerical calculation of three-point branched covers of the projective line}, LMS J.~Comput.~Math.\ \textbf{17} (2014), no.~1, 379--430.

%% \bibitem{LLL}
%% Arjen K.~Lenstra, Hendrik W.~Lenstra, Jr. and L\'azl\'o Lov\' asz, \emph{Factoring polynomials with rational coefficients}, Math. Ann.~\textbf{261} (1982), 513--534.

%\bibitem{Malle} Gunter Malle, \emph{Fields of definition of some three point
%  ramified field extensions}, in \emph{The Grothendieck theory of dessins d'enfants}, London Math.\ Soc.\ Lecture Note Ser., vol.\
%  200, Cambridge University Press, 1994, 147--168.
  
%\bibitem{Monien}
%Hartmut Monien, \emph{The sporadic group $J2$, Hauptmodul and Belyi map}, \texttt{arXiv:1703.05200}.

%\bibitem{Co3}
%Hartmut Monien, \emph{The sporadic group $\mathrm{Co}_3$, Hauptmodul and Belyi map}, \texttt{arXiv:1802.06923}.

%%\bibitem{Shimura1}
%%Goro Shimura, \emph{On some arithmetic properties of modular forms of one and several variables}, Ann.\ of Math.\ \textbf{102} (1975), 491--515.

%%\bibitem{Shimura2}
%% Goro Shimura, \emph{On the derivatives of theta functions and modular forms}, Duke Math.\ J.\ \textbf{44} (1977), 365--387.

%\bibitem{SijslingVoightBirch}
%Jeroen Sijsling and John Voight, \emph{On explicit descent of marked curves and maps}, Res.\ Number Theory 2 (2016), Art.~27, 35 pp.

%\bibitem{SijslingVoight}
%Jeroen Sijsling and John Voight, \emph{On computing \Belyi\ maps}, Publ.~Math.~Besan\c{c}on: Alg\`ebre Th\'eorie Nr.~2014/1, Presses Univ.\ Franche-Comt\'e, Besan\c{c}on, 73--131.

%% \bibitem{vHV2}
%% Mark van Hoeij and Raimundas Vidunas, \emph{Belyi functions for hyperbolic hypergeometric-to-Heun transformations}, J.\ Algebra, vol.~441, 2015, 609--659.

%\bibitem{Weil}
%Andr\'e Weil, \emph{The field of definition of a variety}, Amer.\ J.\ Math.\ \textbf{78} (1956), 509--524.

%% \bibitem{Zograf}
%% Peter Zograf, \emph{Enumeration of Grothendieck's dessins and KP hierarchy}, International Mathematics Research Notices
%%Oxford University Press \textbf{24} (2015), 13533--13544.

%\end{thebibliography}

\end{document}
